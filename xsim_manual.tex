% ----------------------------------------------------------------------------
% the XSIM package
% 
%   eXercise Sheets IMproved
% 
% ----------------------------------------------------------------------------
% Clemens Niederberger
% Web:    https://github.com/cgnieder/xsim
% E-Mail: contact@mychemistry.eu
% ----------------------------------------------------------------------------
% Copyright 2017 Clemens Niederberger
% 
% This work may be distributed and/or modified under the
% conditions of the LaTeX Project Public License, either version 1.3
% of this license or (at your option) any later version.
% The latest version of this license is in
%   http://www.latex-project.org/lppl.txt
% and version 1.3 or later is part of all distributions of LaTeX
% version 2005/12/01 or later.
% 
% This work has the LPPL maintenance status `maintained'.
% 
% The Current Maintainer of this work is Clemens Niederberger.
% ----------------------------------------------------------------------------
% If you have any ideas, questions, suggestions or bugs to report, please
% feel free to contact me.
% ----------------------------------------------------------------------------
\documentclass[load-preamble+]{cnltx-doc}
\usepackage[utf8]{inputenc}
\usepackage{xsim}

\setcnltx{
  package  = {xsim},
%  info     = {comprehensive support for typesetting chemistry documents},
  url      = https://github.com/cgnieder/xsim ,
  authors  = Clemens Niederberger ,
  email    = contact@mychemistry.eu ,
  abstract = {%
    \centering
    \par
  } ,
  quote-format = \small\biolinumLF ,
  add-cmds = {
    blank ,
    collectexercises ,
    collectexercisestype ,
    collectexercisesstop
    DeclareExerciseEnvironmentTemplate ,
    DeclareExerciseGoal ,
    DeclareExerciseHeadingTemplate ,
    DeclareExerciseProperty ,
    DeclareExercisePropertyAlias ,
    DeclareExerciseTableTemplate ,
    DeclareExerciseTranslation ,
    DeclareExerciseTranslations ,
    DeclareExerciseType ,
    ExerciseID ,
    ExerciseGoalName ,
    ExerciseParameterGet ,
    ExercisePropertyGet ,
    ExercisePropertyGetAlias ,
    ExercisePropertyGlobalSave ,
    ExercisePropertyIfSetF ,
    ExercisePropertyIfSetT ,
    ExercisePropertyIfSetTF ,
    ExercisePropertySave ,
    ExercisyType ,
    ForeachExerciseTag ,
    ForEachExerciseTranslation ,
    ForEachPrintedExerciseByType ,
    ForEachPrintedExerciseByID ,
    ForEachUsedExerciseByType ,
    ForEachUsedExerciseByID ,
    GetExerciseAliasProperty ,
    GetExerciseName ,
    GetExerciseParameter ,
    GetExerciseProperty ,
    GlobalSaveExerciseProperty ,
    gradingtable ,
    IfExerciseGoalF ,
    IfExerciseGoalF ,
    IfExerciseGoalTF ,
    IfExercisePropertyExistF ,
    IfExercisePropertyExistT ,
    IfExercisePropertyExistTF ,
    IfExercisePropertySetF ,
    IfExercisePropertySetT ,
    IfExercisePropertySetTF ,
    IfInsideSolutionF ,
    IfInsideSolutionT ,
    IfInsideSolutionTF ,
    NewExerciseCollection ,
    NewExerciseTagging ,
    printcollection ,
    printsolutions ,
    printsolutionstype ,
    SaveExerciseProperty ,
    SetExerciseParameter ,
    SetExerciseParameters ,
    TotalExerciseGoal ,
    TotalExerciseTypeGoal ,
    UseExerciseTags ,
    UseExerciseTagsX ,
    XSIMexpandcode ,
    XSIMifblankT ,
    XSIMifblankF ,
    XSIMifblankTF ,
    XSIMifeqF ,
    XSIMifeqT ,
    XSIMifeqTF ,
    XSIMmixedcase ,
    XSIMputright ,
    xsimsetup ,
    XSIMtranslate
  } ,
  add-envs = { question , solution } ,
  add-silent-cmds = {
  } ,
  index-setup = { othercode = \footnotesize , level = \section } ,
  makeindex-setup = { columns = 2 , columnsep = 1em }
}

\expandafter\def\csname libertine@figurestyle\endcsname{LF}
\usepackage[libertine]{newtxmath}
\expandafter\def\csname libertine@figurestyle\endcsname{OsF}

\usepackage[biblatex]{embrac}
\ChangeEmph{[}[,.02em]{]}[.055em,-.08em]
\ChangeEmph{(}[-.01em,.04em]{)}[.04em,-.05em]

\begin{document}

\section{Licence, Requirements and \textsc{README}}
\license

\xsim\ loads the packages \pkg{expl3}~\cite{bnd:l3kernel},
\pkg{xparse}~\cite{bnd:l3packages}, \pkg{etoolbox}~\cite{pkg:etoolbox},
\pkg{booktabs}~\cite{pkg:booktabs} and
\pkg{translations}~\cite{pkg:translations}.

\section{Motivation and Background}

It has been quite a while since I first published
\pkg{exsheets}~\cite{pkg:exsheets} in Juni 2012.  Since then it has gained a
user base and a little bit of popularity as the number of questions on tex.sx
shows (98~at the time of writing).  User questions, bug reports and feature
requests improved it over the time.  Still it has a version number starting
with a zero which in my versioning system means I still consider it
experimental.

This is due to several facts.  It lacks a few features which I consider
essential for a full version~1.  For one thing it is not possible to have
several kinds of exercises numbered independently.  Using verbatim material
such as listings inside questions and solutions is not possible and the
current workaround isn't that ideal either.  One request which dates back
quite a while now was to have different types of points to questions\ldots

All of those aren't easy to add due to the way \pkg{exsheets} is implemented
right now. As a consequence I wanted to re-implement \pkg{exsheets} for a long
time.  This is what lead to \xsim.  Internally the package works completely
different. It will be the official successor of \pkg{exsheets} which is now
considered obsolete and will only receive bugfix releases any more.

\section{Exercises and Solutions}

\begin{environments}
  \environment{question}[\oarg{properties}]
    Input and typeset an exercise.
  \environment{solution}[\oarg{options}]
    Input and typeset the solution to the exercise of the previous
    \env{question} environment.
\end{environments}

\begin{example}
  \begin{question}
    A first example for a question.
  \end{question}
  \begin{solution}
    A first example for a solution.
  \end{solution}
\end{example}

\section{New Exercise Types}

\section{Exercise Properties}

\section{Exercise Goals}

\section{Exercise Tags}

\section{Collecting Exercises}

\section{Printing Solutions}

\section{Styling the Exercises -- Templates}

\section{Exercise Translations}

\section{Other Commands}

% \blank

\end{document}


%%% Local Variables:
%%% mode: latex
%%% TeX-master: t
%%% End:
