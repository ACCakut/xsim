\documentclass{article}
\usepackage{xsim,lipsum,tcolorbox}

\DeclareExerciseType{exercise}{
  exercise-env = exercise ,
  solution-env = hint ,
  exercise-name = Exercise ,
  solution-name = Hint ,
  exercise-template = subsection* ,
  solution-template = subsection* ,
  counter = question
}

\DeclareExerciseType{problem}{
  exercise-env = problem ,
  solution-env = answer ,
  exercise-name = Problem ,
  solution-name = Answer ,
  exercise-template = tcolorbox ,
  solution-template = tcolorbox
}

\DeclareExerciseTemplate{tcolorbox}
  {%
    \vspace{\baselineskip}
    \tcolorbox[
      colback = red!5!white ,
      colframe = red!75!black ,
      title = \GetExerciseProperty{name}~\GetExerciseProperty{counter}
      \IfExercisePropertySetTF{points}{(\GetExerciseProperty{points})}{}]
  }
  {%
    \endtcolorbox
    \vspace{\baselineskip}
  }

\begin{document}

\begin{question}[points=4,topic=geometry,tags={geometry,triangles},grade=7]
  \lipsum[1]
\begin{verbatim}
foo x^2 y^3
\end{verbatim}
\end{question}
\begin{solution}
  g
\end{solution}

% \end{document}

\begin{question}[subtitle=Foo,points=5,topic=analysis,grade=11]
  \lipsum[2]
\end{question}
\begin{solution}
\end{solution}

% \end{document}

\begin{problem}
  \lipsum[3]
\end{problem}
\begin{answer}
\end{answer}

% \end{document}

\begin{exercise}
  \lipsum[4]
\end{exercise}
\begin{hint}
\end{hint}

% \end{document}

\begin{problem}
  \lipsum[3]
\end{problem}
\begin{answer}
\end{answer}

\begin{question}
  \lipsum[5]
\end{question}
\begin{solution}
\end{solution}

\begin{question}[print=false]
  \lipsum[5]
\end{question}
\begin{solution}
\end{solution}

\end{document}

