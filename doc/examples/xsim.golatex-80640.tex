% http://golatex.de/viewtopic,p,80640.html
\documentclass[fontsize=12pt,parskip=half]{scrartcl}
\usepackage[T1]{fontenc}
\usepackage[ngerman]{babel}

\newlength\breite
\setlength\breite{160mm}
\newlength\hoehe
\setlength\hoehe{80mm}

\usepackage[
  hdivide={3.0cm,\breite,},
  vdivide={2.2cm,,2.2cm}]{geometry}
\usepackage[bitstream-charter]{mathdesign}

\usepackage[babel]{microtype}

\usepackage[inline]{enumitem}

\usepackage{amsmath,siunitx}
\sisetup{
  locale = DE ,
  per-mode = fraction
}

\usepackage{xsim,needspace,tikz,tcolorbox}

\xsimsetup{
  exercise/template = gitter ,
  solution/template = gitter
}

\DeclareExerciseEnvironmentTemplate{gitter}
  {%
    \IfInsideSolutionT{%
      \tcolorbox[
        detach title ,
        title        = Lösung: ,
        before upper = {\tcbtitle\space} ,
        fonttitle    = \bfseries ,
        fontupper    = \large ,
        valign       = center ,
        colframe     = black!50!white ,
        colback      = black!12!white ,
        coltext      = blue ,
        coltitle     = blue ,
        height       = \hoehe ,
        width        = \breite ,
        sharp corners
      ]
    }%
  }
  {%
    \IfInsideSolutionTF
      {\endtcolorbox}
      {%
        \IfSolutionPrintF{%
        \par
        \begin{tikzpicture}[semitransparent]
          \draw[step=4mm, line width=0.15mm, black!60!white] (0,0) grid
            (\breite,\hoehe);
        \end{tikzpicture}
        \par
        }
      }%
  }

\begin{document}

\begin{exercise}
  Für die \textbf{Schülerausgabe} sollen Häuschen (Grid) mit Seitenlänge
  \SI{4}{\mm} gesetzt werden.
\end{exercise}
\begin{solution}
  Hier soll die Lösung stehen: $E=m\cdot c^2$.
\end{solution}

\xsimsetup{solution/print=true}

\begin{exercise}
  Für die \textbf{Lehrerausgabe} sollen statt Häuschen die Lösung in
  z.\,B. einer Box geschrieben werden. Dafür soll die Lösung (bezogen auf
  dieses Beispiel) auch in einer Box mit der exakten Breite \SI{160}{\mm} und
  der exakten Höhe \SI{80}{\mm} gesetzt werden. Weiter soll natürlich die
  Position der Lösungsbox und der Häuschenbox exakt identisch sein.
\end{exercise}
\begin{solution}
  Hier soll die Lösung stehen: $E=m\cdot c^2$.
\end{solution}

\clearpage

\begin{exercise}
  Beschreiben Sie die \textbf{Satzgruppe des Pythagoras}.
\end{exercise}
\begin{solution}
  \begin{align*}
    \left.\begin{aligned}
      c^2   &=a^2+b^2\\
      a^2   &=p\cdot c\;\wedge\; b^2=q\cdot c\\
      h^2   &=p\cdot q
    \end{aligned}\right\}
    \begin{gathered}
      \text{Satzgruppe}\\
      \text{des Pythagoras}
    \end{gathered}
  \end{align*}
\end{solution}

\begin{exercise}
  Geben Sie die Formel der \textbf{gleichförmigen Bewegung} wieder. Gehen Sie
  davon aus, dass Sie 1000~Meter in
  \SI[parse-numbers=false]{3\frac{1}{2}}{Minuten} laufen.
  \begin{enumerate*}[label=\bfseries\alph*)]
    \item Mit welcher Geschwindigkeit legen Sie die Strecke zurück?
    \item Geben Sie die Geschwindigkeit in \si{\km\per\hour} an.
  \end{enumerate*}
  \textbf{Hinweis:} Nehmen Sie zur Berechnung die vereinfachte Form und gehen
  Sie davon aus, dass Ihre Geschwindigkeit konstant ist.
\end{exercise}
\begin{solution}
  \begin{align*}
    s    &=v\cdot t+s_0 \\
    1000 &=v\cdot 210 \\
    v    &=\frac{\SI{1000}{\m}}{\SI{210}{\second}} \\
    v    &=\SI{4.76}{\m\per\second} = \SI{17.1}{\km\per\hour}
  \end{align*}
\end{solution}

\end{document}
