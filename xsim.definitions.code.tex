% ----------------------------------------------------------------------------
% the XSIM package - definitions module
% 
%   eXercise Sheets IMproved
% 
% ----------------------------------------------------------------------------
% Clemens Niederberger
% Web:    https://github.com/cgnieder/xsim
% E-Mail: contact@mychemistry.eu
% ----------------------------------------------------------------------------
% Copyright 2017 Clemens Niederberger
% 
% This work may be distributed and/or modified under the
% conditions of the LaTeX Project Public License, either version 1.3
% of this license or (at your option) any later version.
% The latest version of this license is in
%   http://www.latex-project.org/lppl.txt
% and version 1.3 or later is part of all distributions of LaTeX
% version 2005/12/01 or later.
% 
% This work has the LPPL maintenance status `maintained'.
% 
% The Current Maintainer of this work is Clemens Niederberger.
% ----------------------------------------------------------------------------
% If you have any ideas, questions, suggestions or bugs to report, please
% feel free to contact me.
% ----------------------------------------------------------------------------
\XSIMmodule{definitions}{definition of user commands}

\xsim_load_modules:n {base,exercises,translations,interface}

% ----------------------------------------------------------------------------

\DeclareExerciseProperty* {id}
\DeclareExerciseProperty  {print}
\DeclareExerciseProperty  {name}
\DeclareExerciseProperty  {counter}
\DeclareExerciseProperty  {template}
\DeclareExerciseProperty  {subtitle}
\DeclareExerciseProperty  {points}
\DeclareExerciseProperty  {bonus-points}
\DeclareExerciseProperty  {tags}
\DeclareExerciseProperty  {topic}

% ----------------------------------------------------------------------------

\DeclareExerciseGoal {points}
\DeclareExerciseGoal {bonus-points}

\NewDocumentCommand \printpoints {m}
  {
    \TotalExerciseTypeGoal {#1} {points}
      { \, \XSIMtranslate {point} }
      { \, \XSIMtranslate {points} }
  }

\NewDocumentCommand \printtotalpoints {}
  {
    \TotalExerciseGoal {points}
      { \, \XSIMtranslate {point} }
      { \, \XSIMtranslate {points} }
  }

\NewDocumentCommand \printbonus {m}
  {
    \TotalExerciseTypeGoal {#1} {bonus-points}
      { \, \XSIMtranslate {point} }
      { \, \XSIMtranslate {points} }
  }

\NewDocumentCommand \printtotalbonus {}
  {
    \TotalExerciseGoal {bonus-points}
      { \, \XSIMtranslate {point} }
      { \, \XSIMtranslate {points} }
  }

% ----------------------------------------------------------------------------

\DeclareExerciseType{question}{
  exercise-env      = question ,
  solution-env      = solution ,
  exercise-name     = \XSIMtranslate {question} ,
  solution-name     = \XSIMtranslate {solution} ,
  exercise-template = subsection* ,
  solution-template = subsection*
}

% ----------------------------------------------------------------------------

\DeclareExerciseTemplate{subsection*}
  {
    \subsection*
      {
        \XSIMmixedcase { \GetExerciseProperty {name} } \nobreakspace
        \GetExerciseProperty {counter}
        \IfExercisePropertySetTF{subtitle}
          { ~ { \normalfont \itshape \GetExerciseProperty {subtitle} } }
          {}
    }
    \IfExercisePropertySetTF {points}
      {
        \marginpar
          {
            \IfInsideSolutionTF
              {
                \( \GetExerciseProperty {points} \) \,
                \ExerciseGoalName{points}
                 {\XSIMtranslate{point}}
                 {\XSIMtranslate{points}}
              }
              {
                \rule {1.3cm} {1pt} / \(\GetExerciseProperty {points}
                \IfExercisePropertySetTF {bonus-points}
                  { {}+{} \GetExerciseProperty {bonus-points} }
                  {} \)
              }
          }
      }
      {}
  }
  {}

% ----------------------------------------------------------------------------

\DeclareExerciseTranslations{exercise}{
  Fallback = exercise ,
  English  = exercise ,
  French   = exercice ,
  German   = \"Ubung
}

\DeclareExerciseTranslations{question}{
  Fallback = question ,
  English  = question ,
  French   = question ,
  German   = Aufgabe
}

\DeclareExerciseTranslations{solution}{
  Fallback = solution ,
  English  = solution ,
  French   = solution ,
  German   = L\"osung
}

\DeclareExerciseTranslations{point}{
  Fallback = point ,
  English  = point ,
  French   = point ,
  German   = Punkt
}

\DeclareExerciseTranslations{points}{
  Fallback = points ,
  English  = points ,
  French   = points ,
  German   = Punkte
}

% ----------------------------------------------------------------------------
\tex_endinput:D
